
\section{Outlook}
% Availability of PIRS: in INR, and outside KIT.
% Ongoing work: interface to SERPENT
% Making it open source.

\begin{frame}[fragile]
    \frametitle{Ongoing work}

    \begin{itemize}
        \item Interface to Serpent-2 using multi-physics interface. Toward full-core coupled calculations?
        \item Improvements to SCF interface
        \item clean output
        \item Documentation

            Partly written documentation can be found here:
            \url{http://www.inr.kit.edu/661.php}. Not covered description of
            high-level interfaces and low-level interface to SCF.

        \item make PIRS widely available. In INR it can be found in
            \begin{itemize}
                \item \bashinline/\\sccfs-oe-cn.scc.kit.edu\INR\Gruppen\RPD\Software\PIRS/
                \item \url{https://inrserv02.inr.kit.edu/svn/pirs/}
            \end{itemize}

            For wider distribution we are waiting for a license from KIT legal
            department. They currently propose  GPL license.

    \end{itemize}
\end{frame}

\begin{frame}[fragile]
    \frametitle{Ideas for future work}
        \begin{itemize}
            \item High-level interface to SCF for handling geometry of "unstructured" bundle of rods. Using \bashinline/qhull/ for traingulation 
                  and \bashinline/shapely/ for computing subchannel areas and perimeters.

                  \includegraphics[width=0.3\textwidth]{examples/mini_bwr.pdf}

            \item Integration with other INR tools. Project-oriented? 
                
            \item Development of geometry constructor and high-level interfaces to handle hexagonal geometries.
        \end{itemize}
\end{frame}
