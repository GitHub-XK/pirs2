\documentclass[a4paper]{article}
% generated by Docutils <http://docutils.sourceforge.net/>
\usepackage{fixltx2e} % LaTeX patches, \textsubscript
\usepackage{cmap} % fix search and cut-and-paste in Acrobat
\usepackage{ifthen}
\usepackage[T1]{fontenc}
\usepackage[utf8]{inputenc}
\setcounter{secnumdepth}{0}

%%% Custom LaTeX preamble
% PDF Standard Fonts
\usepackage{mathptmx} % Times
\usepackage[scaled=.90]{helvet}
\usepackage{courier}

%%% User specified packages and stylesheets

%%% Fallback definitions for Docutils-specific commands

% hyperlinks:
\ifthenelse{\isundefined{\hypersetup}}{
  \usepackage[colorlinks=true,linkcolor=blue,urlcolor=blue]{hyperref}
  \urlstyle{same} % normal text font (alternatives: tt, rm, sf)
}{}


%%% Body
\begin{document}


\section{Содержание%
  \label{id1}%
}
%
\begin{quote}
%
\begin{itemize}

\item Введение
%
\begin{quote}
%
\begin{itemize}

\item Что такое PIRS. Определение

\item Для чего он нужен. Маленький пример

\end{itemize}

\end{quote}

\item %
\begin{description}
\item[{Концепция}] \leavevmode %
\begin{itemize}

\item Классы для определения расчетной геометрии

\item Классы-интерфейсы для расчетных кодов

\end{itemize}

\end{description}

\item Реализованные интерфейсы
%
\begin{quote}
%
\begin{itemize}

\item для MCNP

\item для SCF

\end{itemize}

\end{quote}

\item Примеры
%
\begin{quote}
%
\begin{itemize}

\item описание нейтронно-физической модели ТВЭЛа

\item результаты совместного итеративного НФ-ТГ расчета для сборки

\end{itemize}

\end{quote}

\item Заключение
%
\begin{quote}
%
\begin{itemize}

\item Текущая работа

\item Планы на будущее

\end{itemize}

\end{quote}

\end{itemize}

\end{quote}


\section{Введение%
  \label{id2}%
}


\subsection{Что такое PIRS%
  \label{pirs}%
}

PIRS: \textbf{P}ython \textbf{I}nterfaces for \textbf{R}eactor \textbf{S}imulations

Это набор классов для языка программирования Python, описывающих интерфейс для
программ расчета ядерных ректоров.
%
\begin{quote}
%
\begin{itemize}

\item Подготовка входной информации

\item Генерация входных файлов

\item Запуск расчетного кода

\item Чтение результатов

\item Обработка результатов

\end{itemize}

\end{quote}


\subsection{Для чего нужен%
  \label{id3}%
}
%
\begin{quote}
%
\begin{itemize}

\item Связующее звено между программой расчета и интерпретируемым языком программирования.

Облегчает автоматизацию создания входных файлов и чтения результатов.

\item Среда для проведения совместных итерационных расчетов.

Результаты расчета одного кода могут быть использованы для подготовки входных данных
для другого кода.

\end{itemize}

\end{quote}

\end{document}
