

% redefine title color (defined in sphinx style) to black:
%\definecolor{TitleColor}{rgb}{0.0,0.0,0.0}

%\usepackage{snamc2013}
\usepackage{fancyvrb}
\usepackage{graphicx}  % allows inclusion of graphics
\usepackage{booktabs}  % nice rules (thick lines) for tables
\usepackage{microtype} % improves typography for PDF

%\usepackage[breaklinks=true,colorlinks=true,linkcolor=black,citecolor=black]{hyperref}
%\usepackage{hypcap}

%\usepackage{courier}


\title{Python-based framework for coupled MC-TH reactor calculations}

\author[1]{Anton A. Travleev}
\author[1]{Richard Molitor}
\author[1]{Victor Sanchez}

\affil[1]{Institute for Neutron Physics and Reactor Technology (INR), Karlsruhe Institute of Technology,\newline%
Hermann-von-Helmholtz-Platz 1, 76344 Eggenstein-Leopoldshafen}

\abstract{%
We develop a set of Python packages to provide a modern programming interface
to codes used for analysis of nuclear reactors. Currently implemented
interfaces to the Monte Carlo (MC) neutronics code MCNP and thermo-hydraulic
(TH) code SCF allow efficient description of calculation models and provide a
framework for coupled calculations.

In this paper we illustrate how these interfaces can be used to describe a
pin model, and report results of coupled MCNP-SCF calculations performed 
for a PWR fuel assembly, organized by means of the interfaces.
}

\keywords{Python, MCNP, SCF, coupled MC-TH}

